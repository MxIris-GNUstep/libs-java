\chapter{Wrapping Objective-C Libraries}

In this chapter we are going to learn how to use JIGS to expose an
existing Objective-C library to Java.  Technically, we call this
procedure {\sl wrapping} the library.  

\section{Compiling the Wrappers of an Already Configured Library}
We start by reviewing the simplest case: you download the source code
of a library which is already configured to generate java wrappers
when it is compiled.  You can try the discussion out by using any
example in the \texttt{Tools/WrapCreator/Examples/} directory.

\subsection{Creating and Compiling the Wrappers}

In this case, when you type \texttt{make}, the Objective-C will be
compiled as usual; after it is compiled, JIGS will automatically
create a directory called \texttt{JavaWrapper} (or
\texttt{JavaWrapper\_debug} if you are compiling with debug enabled),
and generate inside this directory source code for the wrappers.  To
compile the wrappers, you just need to enter the \texttt{JavaWrapper}
directory, and type \texttt{make} there.

It is quite important to understand that at this point you have two
different entities: the original Objective-C library, and the Java
wrappers for this library.  The two things can be compiled and
installed separately.  To access your library from Objective-C you
only need to have compiled and installed your original library, while
to access it from Java you need to have compiled and installed both
the original library and the wrappers.

\subsection{Structure of the Wrappers}\label{wrapper-structure}

It's not strictly necessary for you to know the structure of the
wrappers to create or use them, so you may safely skip this section.

In the following discussion, we consider wrappers for an imaginary
library called \texttt{libexample.so}.

The wrappers for the library are composed of two parts: a set of Java
classes, and a native Objective-C library, called
\texttt{libexample.A.so} in the example.  This is called after the 
original library (\texttt{libexample.so}) with a \texttt{.A} appended
to the name.

If you go into the \texttt{JavaWrapper} directory, you can inspect
these parts yourself.  There are two main subdirectories in the
\texttt{JavaWrapper} directory, the \texttt{Java} and the
\texttt{Objc} subdirectories.

\subsubsection{The Java Subdirectory}

The \texttt{Java} subdirectory contains the Java classes.  All methods
which are not explicitly marked as \texttt{public} in these classes
are to be considered internal to the JIGS engine and you should not
access them directly (you probably will not be allowed to anyway).

All these classes contain a static initializer of the following form:
\begin{verbatim}
 static
  {
    JIGS.loadLibrary ("example.A");
  }
\end{verbatim}
This will load the native library \texttt{libexample.A.so} as soon as
the first Java class is accessed (\texttt{libexample.A.so} will
implicitly load \texttt{libexample.so}).

Then, most methods are \texttt{native}, as in the following example: 
\begin{verbatim}
  public native void start ();
\end{verbatim}

The implementation of the Java methods marked as \texttt{native} will
be found by the JVM inside the native library \texttt{libexample.A.so}.

\subsubsection{The Objc Subdirectory}

The \texttt{Objc} subdirectory contains the Objective-C library 
\texttt{libexample.A.so}, which provides to Java the implementation 
of all the \texttt{native} methods in the Java classes.  The
implementation of these methods will call the corresponding methods in
the (pure Objective-C) \texttt{libexample.so} library, by using the
JIGS core engine to catch exceptions, convert arguments and return
types, etc.

\section{Turning on Generation of Java Wrappers}
We assume that your library is an objective-C library which is managed
using the GNUstep make package.

To turn on the generation of the java wrappers for your objective-C
library, all you need to do is to add the line
\begin{verbatim}
include $(GNUSTEP_MAKEFILES)/java-wrapper.make
\end{verbatim}
after including \texttt{library.make}, that is, after the line
\begin{verbatim}
include $(GNUSTEP_MAKEFILES)/library.make
\end{verbatim}

When you now compile your library (by typing \texttt{make}), JIGS will
try to generate the java wrappers.  In this way, you can simply turn
off and on generation of the java wrappers by commenting and
uncommenting the inclusion of \texttt{java-wrapper.make}.  To
correctly generate the java wrappers, you will actually need to
instruct JIGS about what classes and methods you want exposed, and
how.  This will be discussed in detail in the following sections.

\section{Specifying the Main Library Header}
JIGS automatically parses your library headers to determine the exact
method declarations of the methods you want to expose.  

You need to specify a single header file for your library which
includes all the header files for your library.  JIGS tries to
determine it in the following way: it takes the name of your library 
(the one you set by setting the \texttt{LIBRARY\_NAME}), removes 
the trailing \texttt{lib}, then looks for a header file with that name 
in the directory \texttt{HEADER\_FILES\_DIR}. 

For example, if the GNUmakefile for your library is the following:

\begin{verbatim}
include $(GNUSTEP_MAKEFILES)/common.make

LIBRARY_NAME = libGame

libGame_OBJC_FILES = Alien.m Fuel.m SpaceShip.m Sprite.m

libGame_HEADER_FILES_DIR = Headers
libGame_HEADER_FILES_INSTALL_DIR = /Game
libGame_HEADER_FILES = Alien.h Game.h Fuel.h SpaceShip.h Sprite.h

include $(GNUSTEP_MAKEFILES)/library.make

# Comment the following line to disable java wrappers
include $(GNUSTEP_MAKEFILES)/java-wrapper.make
\end{verbatim}
%$ [fool emac's buggy TeX mode]

Then your library name is \texttt{libGame}, which is shortened to
\texttt{Game}.  The header file dir is \texttt{Headers}, so that 
by default JIGS will (try to) parse the file \texttt{Headers/Game.h}
as the library header file.  This header file should include all the
header files which might be need to compile code against your library.

If the header which JIGS needs to parse is not correctly determined in
this way, you can override it by adding a \texttt{GNUmakefile.wrapper}
to your project, and defining \texttt{WRAPPER\_HEADER} manually there.
For example, if the header you want to be parsed is called
\texttt{game.h} rather than \texttt{Game.h}, you can add the following 
\texttt{GNUmakefile.wrapper} to fix this: 
\begin{verbatim}
WRAPPER_HEADER = Headers/game.h
\end{verbatim}
The \texttt{GNUmakefile.wrapper} is automatically included (if present) 
by JIGS at the correct stage.

The other variables you may want to set in the
\texttt{GNUmakefile.wrapper} are \texttt{JIGS\_FILE} and
\texttt{WRAPPER\_DIR}.  We will talk about these in the next section.

\section{The Wrapper Directory}
JIGS generates full code for the java wrappers into a specific
directory, which we call the wrapper directory.  By default, this
directory is called \texttt{JavaWrapper} (or
\texttt{JavaWrapper\_debug} if you are compiling with debugging
enabled), but you can change the name of the directory by simply
setting the \texttt{WRAPPER\_DIR} variable in your
\texttt{GNUmakefile.wrapper}.  This is not very common.

To determine if the wrappers are to be rebuilt or not, JIGS keeps a
stamp file in the wrapper directory.

Everything which is inside the wrapper directory is automatically
generated -- this means that
\begin{enumerate}
\item you can always remove the wrapper directory without loosing anything, 
since you can always regenerate the wrappers;
\item you should not edit or modify directly the contents of the wrapper 
directory, because when the wrappers are recreated, the content of the 
directory is destroyed.
\end{enumerate}

\section{The .jigs File}

\subsection{Determining which .jigs file to read}
After running the preprocessor on the header file, JIGS will read a
configuration file to know which wrappers to generate, and how.  By
default, the file is called \texttt{\$(LIBRARY\_NAME).jigs}.  In the
case described before, this file would be called
\texttt{libGame.jigs}.  You can override this choice by setting the
\texttt{JIGS\_FILE} variable in your \texttt{GNUmakefile.wrapper}, 
but you shouldn't usually need to.

\subsection{Overview of the .jigs file}
The \texttt{.jigs} file is the main file from which you can control
the way the wrapping is done.  In this file, you need to list all the
classes of your library which you want to expose, and for each class,
which methods.  The \texttt{.jigs} file is in a format called 
\emph{property list} -- please refer to the GNUstep Base Library tutorials 
and documentation if you need more help on property lists.  The
\texttt{.jigs} file is a big dictionary.  Each entry in the dictionary 
is to be thought of as of a different section; the key to the section
is the section name, the value is an array or a dictionary containing
the information referring to that section.  For example, we have the
\texttt{classes} section (containing an array of classes to be wrapped), the
\texttt{types} section (containing a dictionary of mapping of custom C 
types to already known C types), the \texttt{enumerations} section
(containing an array listing enumeration types), the \texttt{method
name mapping} section (containing a dictionary of mappings from
objective-C method names to Java method names, to be used for the
whole library).  Here is an example:
\begin{verbatim}
{
  classes = (
    {
      "java name" = "gnu.gnustep.GreatGameLibrary.GGLSprite";
      "objective-c name" = GGLSprite;
      "instance methods" = (speed, "setSpeed:");
      initializers = (init);
     }
   );
  
  "method name mapping" = {
    speed = getSpeed;
  };
}
\end{verbatim}

In this example, we only have the \texttt{classes} and \texttt{method
name mapping} sections.  We now examine in detail each section of the
\texttt{.jigs} file.

\subsection{The Classes Section}
Probably the most important section in the file is the
\texttt{classes} section, where you list all the classes you want to
be wrapped, and how.  This section is an array, each entry in the
array being a different class to be wrapped.  Here is an example 
of this section:
\begin{verbatim}
classes = (
  {
    "java name" = "gnu.gnustep.GreatGameLibrary.GGLScore";
    "objective-c name" = "GGLScore";

    "initializers" = ("init"); 
    "instance methods" = ("setPoints:", "points");
    "class methods" = ("bestScore", "bestTenScores");
  },

  {
    "java name" = "gnu.gnustep.GreatGameLibrary.GGLPlayer";
    "objective-c name" = "GGLPlayer";

    "class methods" = ("computerPlayer");
    "instance methods" = ("name", "score", "addPoints:");
    "initializers" = ("init", "initWithName:");                 
  }
);
\end{verbatim}
In this example, we want to generate wrappers for two classes:
\texttt{GGLScore} and \texttt{GGLPlayer}.  Information about each 
class is represented by a dictionary.  Any entry in this dictionary
(except the `\texttt{java name}' entry) can be omitted, in which case
a default value will be used.  We briefly now review the use of each
entry.

\subsubsection{java name}
This is a string -- the name that is to be used in Java for the class
you want to wrap.  This is the only compulsory entry for each class.
You need to specify the full Java name of the class, such as
\begin{verbatim}
"java name" = "gnu.gnustep.base.NSArray";
\end{verbatim}
You can decide to use a different name in Java than the name the class
has in Objective-C.  It's generally better to avoid this, but it
should work fine if you ever need to do it.  

\subsubsection{objective-c name}
This is a string -- the Objective-C name of the class you want to
wrap, such as 
\begin{verbatim}
"objective-c name" = "NSArray";
\end{verbatim}
If this entry is not present, it is guessed by taking the last part of
\texttt{java name}.  For example, if \texttt{java name} is 
\texttt{gnu.gnustep.base.NSArray}, then JIGS will use \texttt{NSArray} 
as Objective-C name if you don't specify an Objective-C name.

\subsubsection{initializers}
This is an array -- the list of \texttt{init} methods you want to
wrap, such as
\begin{verbatim}
"initializers" = ("init", "initWithName:", "initWithArray:", "initWithName:array:");
\end{verbatim}
Each method in the \texttt{initializers} list will be exposed as a
constructor.  The Objective-C name of each initializer is discarded 
in Java, so you must make sure that these methods have all different 
signatures.

\subsubsection{class methods}
This is an array -- the list of class methods you want to wrap, such as 
\begin{verbatim}
"class methods" = (
  "contentRectForFrameRect:styleMask:", 
  "frameRectForContentRect:styleMask:", 
  "minFrameWidthWithTitle:styleMask:",
  "removeFrameUsingName:");
\end{verbatim}
Each of these methods is exposed as a static method in Java.  Unless
you use a method mapping (explained below), the Java name of each of
these methods will be obtained by removing from the Objective-C name
everything which comes after the first `\texttt{:}'.  For example, the
methods in the example would be wrapped as:
\begin{verbatim}
public static native NSRect contentRectForFrameRect (NSRect arg0, long arg1);
public static native NSRect frameRectForContentRect (NSRect arg0, long arg1);
public static native NSRect minFrameWidthWithTitle (String arg0, long arg1);
public static native void removeFrameUsingName (String arg0);
\end{verbatim}

Please make sure you do not include any methods of the \texttt{new}
family in the \texttt{class methods} list.  \texttt{new} methods can not
be wrapped because they allocate and return non-autoreleased objects
-- it should not be a problem anyway, because if you need that
constructor in Java, you can get it by simply wrapping the
corresponding \texttt{init} method.

\subsubsection{instance methods}
This is an array -- the list of instance methods you want to wrap,
such as 
\begin{verbatim}
"instance methods" = (
  "arguments",
  "environment",
  "hostName",
  "processName",
  "globallyUniqueString",
  "setProcessName:");
\end{verbatim}
Each of these methods is exposed as an instance method in Java.  The
Java names for these methods are obtained in the same way as for class
methods.

Make sure you do not include any method of the \texttt{copy} and 
\texttt{mutableCopy} family in this list.  Methods of these families 
can not be wrapped like the other ones because they return
newly-created non-autoreleased objects.  It is possible that in the
future JIGS could provide a way to expose these methods; at present,
you can not expose them -- except the ones exposed by \texttt{NSObject}: 
\texttt{NSObject} exposes \texttt{copy} (as \texttt{clone}) and 
\texttt{mutableCopy} (as \texttt{mutableClone}) to Java, so if your 
Objective-C class implements \texttt{copy} and \texttt{mutableCopy},
they will automatically be made available to Java by the
\texttt{NSObject} wrapper without any need to do anything.

\subsubsection{method declarations}
This is an array -- a list of explicit method declarations which you
can use to override the declarations found by the wrapper tool in the
header file if you think the wrapper tool did not use the correct
declaration -- this usually happens only if in your library the same
method is declared in different classes or protocols with different
arguments or return types (this case is infrequent, but it happens
sometime) -- in this case the wrapper tool is likely not to pick up
the correct declaration, because it is designed for the standard case,
when the same signature is used across all classes.  By adding the
correct declaration for your class in the \texttt{method declaration}
section, you can make sure that the declaration you want is used.  For
example, in the gui library the following explicit declaration is used
when wrapping
\texttt{NSBox}:
\begin{verbatim}
"method declarations" = (
   "- (void) setContentView: (NSView*)aView");
\end{verbatim}
This is because \texttt{NSScrollView} uses a conflicting declaration
for this method, taking a \texttt{NSClipView *} argument.

Please note that, while in Objective-C these problems are harmless,
and usually the worst it can happen is that the compiler warns you at
compile time, after you expose your methods to Java these problems
become first-rate problems, because in Java casts are dramatic and
your program can crash if a method expecting a \texttt{NSClipView}
argument receives a generic \texttt{NSView} instead.

\subsubsection{hardcoded constants}
This is an array -- a list of declaration of constants which will be
added as static constants to the Java class, as in the following
example:
\begin{verbatim}
"hardcoded constants" = ("int NotFound = 0x7fffffff");
\end{verbatim}
This \texttt{hardcoded constants} section will generate the following
code in the Java class:
\begin{verbatim}
public static int NotFound = 0x7fffffff;
\end{verbatim}
This example is taken from the wrapping instructions for
\texttt{NSArray}; it allows the Java programmer to access this
constant simply as \texttt{NSArray.NotFound}.  The value of this
constant is of course the value of the C constant \texttt{NSNotFound}
which is returned by many GNUstep Base Library methods when something
is not found.

If your Objective-C library defines some constants, and some methods
which you expose to Java return or accept these constants, you may
want to make these constants available to Java as well by using the
\texttt{hardcoded constants} section.

\subsection{Method Name Mapping}
This section is a dictionary -- mapping Objective-C method names 
to Java method names, as in the following example:
\begin{verbatim}
"method name mapping" = {
  "null" = "nullValue";
  "addTimer:forMode:" = "addTimerForMode";
};
\end{verbatim}
Try to avoid method mappings, because they are confusing!  But
sometimes you can't do without, as in the case of \texttt{null} which
can't be exposed with this name to Java because \texttt{null} is a
reserved keyword in Java.  In this case, you use the method mapping.
Please note that JIGS needs to keep a global table of method name
mappings while running in order to map selectors between Java and
Objective-C; for this reason, you need to use the same mappings in all
the classes.

\subsection{Prerequisite Libraries}
Sometimes, the wrappers for your library need the wrappers of some
other GNUstep Objective-C library to work correctly.  For example, you
can't use the wrappers for the GNUstep Gui Library without the
wrappers for the GNUstep Base Library.  If this is the case for your
library, you need to add a \texttt{prerequisite libraries} section,
containing a list of GNUstep libraries whose wrappers you want to be
loaded at run-time when the wrappers for your library are initialized.
For example, in \texttt{libgnustep-gui.jigs} there is
\begin{verbatim}
"prerequisite libraries" = ("gnustep-base");
\end{verbatim}
You will probably usually need at least this line.  Please note that
the library is specified using the simplest possible name; JIGS at
run-time will add the correct prefixes (\texttt{lib}) and suffixes
(things like \texttt{\_d} for debug version -- if needed -- and
\texttt{.A} for the wrappers) to load in the library.

\subsection{Types}
The wrapper tool can parse and manage a certain set of C types.  For
example, upon finding \texttt{BOOL} in the Objective-C headers, the
wrapper tool knows how to manage it.

In the \texttt{types} section you can specify a list of C types
specific to your library, and for each of them, a corresponding
(known) C type.  The wrapper tool will know it has to convert to/from
Java the first C type in the same way as it does with the second.  For
example, in the \texttt{libgnustep-base.jigs} file, there is the
following line:
\begin{verbatim}
types = { NSTimeInterval = double; };
\end{verbatim}
This simply tells to the wrapper tool that whenever it finds an argument 
of type \texttt{NSTimeInterval} in an Objective-C method declaration, it 
has to treat it in the same way as if it were a \texttt{double}.

\subsection{Enumerations}
The \texttt{enumerations} section is simply a short-hand to make
wrapping enumerations simpler.  This section contains an array, in
which you can list all the new enumeration types in your Objective-C
library that you want to be exposed:
\begin{verbatim}
enumerations = (NSTitlePosition);
\end{verbatim}
The enumeration types are exposed in the same way as \texttt{int}s
are.  In other words, this is completely equivalent to putting
\texttt{NSTitlePosition} in the \texttt{types} section mapping it to
the \texttt{int} type:
\begin{verbatim}
types = { NSTitlePosition = int; };
\end{verbatim}
Using the \texttt{enumerations} section rather than mixing
enumerations with the other types makes simply the wrapping file
clearer to read, and lets you type less.

Usually, after declaring an enumeration in the \texttt{enumerations} 
section, you need to add the various values as hardcoded constants 
in one of your Java classes.  For example, in the GNUstep Gui 
Library, after having declared \texttt{NSTitlePosition} as an 
enumeration, the following entry is added to the \texttt{NSBox} 
class entry:
\begin{verbatim}
"hardcoded constants" = (
   "int NoTitle = 0",
   "int AboveTop = 1",
   "int AtTop = 2",
   "int BelowTop = 3",
   "int AboveBottom = 4",
   "int AtBottom = 5",
   "int BelowBottom = 6");
\end{verbatim}
This will generate the following code in the Java \texttt{NSBox} class:
\begin{verbatim}
/* Constants */

public static int NoTitle = 0;
public static int AboveTop = 1;
public static int AtTop = 2;
public static int BelowTop = 3;
public static int AboveBottom = 4;
public static int AtBottom = 5;
public static int BelowBottom = 6;
\end{verbatim}
This makes sure the various values for \texttt{NSTitlePosition}
arguments are available from Java quite simply as \texttt{NSBox.NoTitle} 
etc.

\section{Useful Make Options when Creating Wrappers}

At this point, typing \texttt{make} should be enough to generate
wrappers for your library.  

\subsection{Silencing the Wrapper Generator}

By default, wrapper generation is quite verbose -- it tells you
exactly what it is doing -- for example, it lists all the classes and
methods which are being wrapped.  If you find this annoying, you can
turn it off by compiling with the \texttt{verbose=no} option, as in:
\begin{verbatim}
make verbose=no
\end{verbatim}

\subsection{Enabling Debugging}

If you are compiling your library with debugging enabled  -- ie, typing
\begin{verbatim}
make debug=yes
\end{verbatim}
the wrapper library will also be generated with debugging enabled.
This means that the GNUmakefiles for the wrapping library will
always automatically use \texttt{debug=yes} when the wrapping library
is compiled.  The wrapping code itself will be generated in the
\texttt{JavaWrapper\_debug} directory rather than in the \texttt{JavaWrapper}, 
to allow you to keep the two versions.

\subsection{Debugging or Non-Debugging}

When Objective-C GNUstep libraries are loaded from Java using JIGS,
all the libraries must be of the same type -- or debugging libraries
or non-debugging libraries.  

In general, the simplest way you can manage this issue is by
consistently using only libraries of one of the two kind on your
system; ie, compile everything (from the GNUstep Base Library up) with
`\texttt{make}', or compile everything with `\texttt{make debug=yes}'.

If you don't, then you need to read carefully the following part.

\subsubsection{The Fine Prints About Debugging}

JIGS will decide whether to use debugging or non-debugging libraries
at run-time, when it is first initialized from Java.  Debugging
libraries are easy to tell because the name ends with \texttt{\_d}.
For example, the core JIGS library is
\texttt{libgnustep-java}; if you compile it without debugging, it will 
be in the file 
\begin{verbatim}
libgnustep-java.so
\end{verbatim}
but if you compile it with debugging, it will be in the file 
\begin{verbatim}
libgnustep-java_d.so
\end{verbatim}
Because the name is different, you can have both versions on your
system.  JIGS will choose to load debugging or non-debugging libraries
in the following way:
\begin{enumerate}
\item If you have only one version of JIGS installed (ie, only the debugging 
or the non-debugging, but not both), there is no choice -- JIGS will
load libraries of the same kind as itself.  This means if you have a
debugging version of JIGS installed, you will need to have a version
of all libraries you need with debugging enabled.  If you only have a
non-debugging version of JIGS installed, you will need to have a
version of all libraries you need without debugging.
\item If you have both version of JIGS (the debugging and the non-debugging), 
JIGS will decide at run-time which version of itself to use (and
consequently, which version of all other libraries to use).  By
default, it will load the non-debugging version.  This is the default
because the idea is that using a debugger to debug libraries when
accessed from Java in this way is so difficult that most people will
not want to do it (I actually don't know how to do it myself).  But
you can override the default by setting the \texttt{JIGS\_DEBUG}
environment variable.  Setting it to \texttt{YES} or \texttt{yes} will
give priority to the debugging libraries.
\end{enumerate}
Once JIGS has chosen an option, it will follow it consistently.  For
example, if it decided to load debugging libraries, it will ignore all
non-debugging libraries.

\section{Including Special Code in the Wrapper GNUmakefiles}

This section describes a feature which is usually not needed, so you
may skip it at a first reading.  As a prerequisite, you probably need
to have read the section \ref{wrapper-structure} to make any use of
this section.

It is possible that you may need/want to insert special code in the
GNUmakefile.preamble and GNUmakefile.postamble for the wrappers.  For
example, some special flags required to compile an Objective-C library
(the Objective-C wrapper library for your library) against your
Objective-C library.

JIGS itself by default does not generate GNUmakefile.preamble and
GNUmakefile.postamble at all; but it will look in the library
directory for the files 
\begin{verbatim}
GNUmakefile.wrapper.objc.preamble
GNUmakefile.wrapper.objc.postamble
GNUmakefile.wrapper.java.preamble
GNUmakefile.wrapper.java.postamble
\end{verbatim}
and, if any of them is present, copy them into the \texttt{JavaWrapper} 
directory, in the following way:
\begin{verbatim}
GNUmakefile.wrapper.objc.preamble --> JavaWrapper/Objc/GNUmakefile.preamble
GNUmakefile.wrapper.objc.postamble --> JavaWrapper/Objc/GNUmakefile.postamble
GNUmakefile.wrapper.java.preamble --> JavaWrapper/Java/GNUmakefile.preamble
GNUmakefile.wrapper.java.postamble --> JavaWrapper/Java/GNUmakefile.postamble
\end{verbatim}

In this way, by adding one of these files to your library's directory,
you can have it copied into the generated wrappers in the appropriate
place.
